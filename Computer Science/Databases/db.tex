\documentclass{article}
\setlength{\parindent}{0pt} 
\usepackage[
top    = 2.50cm,% presumably you don't want it to be 0pt as well?
bottom = 2.50cm,
left   = 2cm,
right  = 2cm,
marginparsep = 0pt,
marginparwidth=0pt,
]{geometry}
\usepackage[T1]{fontenc}       % Postscript encoding

\usepackage{titlesec}

\usepackage{multicol}

\usepackage[official]{eurosym}

\usepackage{fancyvrb,newverbs,xcolor}

\definecolor{cverbbg}{gray}{0.93}

\newenvironment{cverbatim}
{\SaveVerbatim{cverb}}
{\endSaveVerbatim
	\flushleft\fboxrule=0pt\fboxsep=.5em
	\colorbox{cverbbg}{\BUseVerbatim{cverb}}%
	\endflushleft
}
\newenvironment{lcverbatim}
{\SaveVerbatim{cverb}}
{\endSaveVerbatim
	\flushleft\fboxrule=0pt\fboxsep=.5em
	\colorbox{cverbbg}{%
		\makebox[\dimexpr\linewidth-2\fboxsep][l]{\BUseVerbatim{cverb}}%
	}
	\endflushleft
}

\newcommand{\ctexttt}[1]{\colorbox{cverbbg}{\texttt{#1}}}
\newverbcommand{\cverb}
{\setbox\verbbox\hbox\bgroup}
  {\egroup\colorbox{cverbbg}{\box\verbbox}}

\usepackage{csvsimple}

\usepackage{xcolor}

\usepackage{graphicx} %

\usepackage{lmodern}           % Latin modern for main font

\usepackage{tgadventor}            % Helvetica for headings

	% Title Formats
\definecolor{gray75}{gray}{0.75}
\newcommand{\hsp}{\hspace{20pt}}
\titleformat{\section}[hang]{\normalfont\sffamily\LARGE\bfseries\flushright}{\thesection\hsp\textcolor{gray75}{|}\hsp}{0em}{}
\titleformat{\subsection}[hang]{\normalfont\sffamily\Large\bfseries}{}{0em}{}
\titleformat{\subsubsection}[hang]{\normalfont\sffamily\large\bfseries\flushright}{}{0em}{}

% Title spacings
\titlespacing*{\section} {0pt}{3.5ex plus 1ex minus .2ex}{25pt}
\titlespacing*{\subsection} {0pt}{3.25ex plus 1ex minus .2ex}{10pt}
\titlespacing*{\subsubsection} {10pt}{2ex plus 1ex minus .2ex}{10pt}


\usepackage{fancyhdr}

\cfoot{\thepage}
\rfoot{\footnotesize\scshape Work in Progress}	

\usepackage{titlesec}


%Crow's foot notation table
\usepackage{booktabs}
\usepackage{array}
\renewcommand{\arraystretch}{1.4}
\usepackage{tikz}
\tikzset{
	zig zag to/.style={
		to path={(\tikztostart) -| ($(\tikztostart)!#1!(\tikztotarget)$) |- (\tikztotarget) \tikztonodes}
	},
	zig zag to/.default=0.5,
	one to one/.style={
		one-one, zig zag to
	},    
	one to none/.style={
		one-, zig zag to
	},    
	oone to none/.style={
		oone-, zig zag to
	},    
	omany to none/.style={
		omany-, zig zag to
	},    
	one to many/.style={
		one-crow's foot, zig zag to,
	},
	one to omany/.style={
		one-omany, zig zag to
	},      
	many to one/.style={
		crow's foot-one, zig zag to
	},
	many to many/.style={
		crow's foot-crow's foot, zig zag to
	}, 
	many to none/.style={ 
		crow's foot-, zig zag to 
	},
}    
\makeatletter
\pgfarrowsdeclare{crow's foot}{crow's foot}
{
	\pgfarrowsleftextend{+-.5\pgflinewidth}%
	\pgfarrowsrightextend{+.5\pgflinewidth}%
}
{
	\pgfutil@tempdima=0.6pt%
	%\advance\pgfutil@tempdima by.25\pgflinewidth%
	\pgfsetdash{}{+0pt}%
	\pgfsetmiterjoin%
	\pgfpathmoveto{\pgfqpoint{0pt}{-9\pgfutil@tempdima}}%
	\pgfpathlineto{\pgfqpoint{-13\pgfutil@tempdima}{0pt}}%
	\pgfpathlineto{\pgfqpoint{0pt}{9\pgfutil@tempdima}}%
	\pgfpathmoveto{\pgfqpoint{0\pgfutil@tempdima}{0\pgfutil@tempdima}}%
	\pgfpathmoveto{\pgfqpoint{-8pt}{-6pt}}% 
	\pgfpathlineto{\pgfqpoint{-8pt}{-6pt}}%  
	\pgfpathlineto{\pgfqpoint{-8pt}{6pt}}% 
	\pgfusepathqstroke%
}

\pgfarrowsdeclare{omany}{omany}
{
	\pgfarrowsleftextend{+-.5\pgflinewidth}%
	\pgfarrowsrightextend{+.5\pgflinewidth}%
}
{
	\pgfutil@tempdima=0.6pt%
	%\advance\pgfutil@tempdima by.25\pgflinewidth%
	\pgfsetdash{}{+0pt}%
	\pgfsetmiterjoin%
	\pgfpathmoveto{\pgfqpoint{0pt}{-9\pgfutil@tempdima}}%
	\pgfpathlineto{\pgfqpoint{-13\pgfutil@tempdima}{0pt}}%
	\pgfpathlineto{\pgfqpoint{0pt}{9\pgfutil@tempdima}}%
	\pgfpathmoveto{\pgfqpoint{0\pgfutil@tempdima}{0\pgfutil@tempdima}}%  
	\pgfpathmoveto{\pgfqpoint{0\pgfutil@tempdima}{0\pgfutil@tempdima}}%
	\pgfpathmoveto{\pgfqpoint{-6pt}{-6pt}}% 
	\pgfpathcircle{\pgfpoint{-11.5pt}{0}} {3.5pt}
	\pgfusepathqstroke%
}

\pgfarrowsdeclare{oone}{oone}
{
	\pgfarrowsleftextend{+-.5\pgflinewidth}%
	\pgfarrowsrightextend{+.5\pgflinewidth}%
}
{
	\pgfutil@tempdima=0.6pt%
	%\advance\pgfutil@tempdima by.25\pgflinewidth%
	\pgfsetdash{}{+0pt}%
	\pgfsetmiterjoin%
	\pgfpathmoveto{\pgfqpoint{0\pgfutil@tempdima}{0\pgfutil@tempdima}}%
	\pgfpathmoveto{\pgfqpoint{-6pt}{-6pt}}% 
	\pgfpathlineto{\pgfqpoint{-6pt}{-6pt}}%  
	\pgfpathlineto{\pgfqpoint{-6pt}{6pt}}% 
	\pgfpathcircle{\pgfpoint{-11.5pt}{0}} {3.5pt}
	\pgfusepathqstroke%
}

\pgfarrowsdeclare{one}{one}
{
	\pgfarrowsleftextend{+-.5\pgflinewidth}%
	\pgfarrowsrightextend{+.5\pgflinewidth}%
}
{
	\pgfutil@tempdima=0.6pt%
	%\advance\pgfutil@tempdima by.25\pgflinewidth%
	\pgfsetdash{}{+0pt}%
	\pgfsetmiterjoin%
	\pgfpathmoveto{\pgfqpoint{0\pgfutil@tempdima}{0\pgfutil@tempdima}}%
	\pgfpathmoveto{\pgfqpoint{-6pt}{-6pt}}% 
	\pgfpathlineto{\pgfqpoint{-6pt}{-6pt}}%  
	\pgfpathlineto{\pgfqpoint{-6pt}{6pt}}% 
	\pgfpathmoveto{\pgfqpoint{0\pgfutil@tempdima}{0\pgfutil@tempdima}}%
	\pgfpathmoveto{\pgfqpoint{-8pt}{-6pt}}% 
	\pgfpathlineto{\pgfqpoint{-8pt}{-6pt}}%  
	\pgfpathlineto{\pgfqpoint{-8pt}{6pt}}%    
	\pgfusepathqstroke%
}

\begin{document}
	\section{Introduction to Databases}
	\subsection{Database Nomenclature}
	\begin{center}
			\begin{tabular}{|c|c|}
				\hline&\\
			\textbf{Files}     & \textbf{Databases}\\&\\ \hline&\\
				File       & Table            \\&\\
				Records    & Tuples / Records \\&\\
				Fields     & Attributes       \\&\\
				Data Items & Values           \\&\\
				Key Field  & Primary Key      \\&\\ \hline
			\end{tabular}
	\end{center}

	
	\subsection{Filing systems}
		Consider the existence of two files:
		\begin{itemize}
			\item{\textbf{Employee File}}
				\subitem{This would contain items such as ID number, name, surname, address, email and date of birth.}
			\item\textbf{{Salary File}}
				\subitem{This would contain items such as hours worked, rates of pay, overtime done, sick leave periods and vacation hours.}
		\end{itemize}
	
		To obtain a link between these two files (\textit{i.e binding a salary file to an employee file}) one must repeat all the data manually creating redundancies, duplications, inconsistencies, losses of data integrity, decentralization and, loss of security.
		
	\section{Database systems}
		To completely omit filing systems, database systems are used instead. These avoid data duplication, redundancy, inconsistency, loss of integrity, insecurity and decentralization.\\
		
		The \textit{IDE} of databases are called \textbf{DBMS}\footnote{Data base management system}.The various types of databases are: 
		\begin{itemize}
			\item{Flat File}
			\item{Relational}
			\item{Hierarchical}
			\item{Network}
			\item{Object Oriented}
		\end{itemize}
	\subsection{Flat File Database}
	Below you can find the \cverb|CSV| version of a flat file database and its visual representation.
\begin{lcverbatim}
Order, AccountNumber, Name, Address, OrderDate, ItemName, ItemQty, ItemPrice, OrderTotal
12, 3, Peppi, B'kara, 10/2/2021, Pizza, 7, 4.20, 29.40
12, 3, Peppi, B'kara, 10/2/2021, Pasta, 1, 6.90, 6.90 
13, 4, Cikku, B'bugia, 11/3/2021, Pomodoro, 3, 1.00, 3.00
\end{lcverbatim}
\begin{center}
		\begin{tabular}{|c|c|c|c|c|c|c|c|c|}%
		\hline
		\bfseries Order & \bfseries A/C no. & \bfseries Name &\bfseries City &\bfseries Date & \bfseries Item &\bfseries Quantity &\bfseries Price E.A &\bfseries Total 
		\csvreader[head to column names]{flat_file.csv}{}% use head of csv as column names
		{\\\hline\Order & \AccountNumber &  \Name & \Address & \OrderDate & \ItemName & \ItemQty & \euro\ItemPrice & \euro\OrderTotal}
		\\\hline
	\end{tabular}
\end{center}


	\subsection{Relational Database}
	The three types of relations can be denoted with the notation that is referred to as \textit{Crow's Foot} notation. Edgar F.	Codd invented a set of related tables using normalization rules which include 3 forms: $1\textsuperscript{st}$, $2\textsuperscript{nd}$ and $3\textsuperscript{rd}$ normal forms.
	\begin{center}
		
\begin{minipage}{\textwidth}% Minipage has width exactly the same as the text block
	\centering% Centers the contents of the minipage
	\begin{tabular}{lc}
		\toprule
		Jargon & Notation\\
		\midrule
		one to none &
		\tikz{\draw[one to none] (0,0) -- ++(1.5,0);}\\
		one to many & \tikz{\draw[one to many] (0,0) -- ++(1.5,0);}\\
		many to none & \tikz{\draw[many to none] (0,0) -- ++(1.5,0);}\\
		many to one & \tikz{\draw[many to one] (0,0) -- ++(1.5,0);}\\ 
		many to many &\tikz{\draw[many to many] (0,0) -- ++(1.5,0);}\\
		\bottomrule

	\end{tabular}
	\medskip	
	\textcolor{gray}{Crow's foot notation.}% Your caption goes here.
\end{minipage}
	\end{center}
%FULL CBN
%	\begin{tabular}{lcc}
%		\toprule
%		Notation & IEM & CHEN \\
%		\midrule
%		one to none &
%		\tikz{\draw[one to none] (0,0) -- ++(1.5,0);} & 1:0\\
%		one to one & \tikz{\draw[one to one] (0,0) -- ++(1.5,0);} & 1:1\\
%		one to many & \tikz{\draw[one to many] (0,0) -- ++(1.5,0);} & 1:n\\
%		many to none & \tikz{\draw[many to none] (0,0) -- ++(1.5,0);} & n:0\\
%		one to (none or one or many) &\tikz{\draw[one to omany] (0,0) -- ++(1.5,0);} & \\ 
%		many to one & \tikz{\draw[many to one] (0,0) -- ++(1.5,0);} & n:1\\ 
%		many to many &\tikz{\draw[many to many] (0,0) -- ++(1.5,0);} & n:m\\
%		(none or one or many) to none &\tikz{\draw[omany to none] (0,0) -- ++(1.5,0);} & \\
%		(none or one) to none &\tikz{\draw[oone to none] (0,0) -- ++(1.5,0);} & \\
%		\bottomrule
%	\end{tabular}

\subsubsection{First normal form}
In this form, the multi-variable categories should be exported in another, separate table.

\begin{center}
	\begin{tabular}{|c|c|c|c|c|c|c|c|c|}%
		\hline
		\bfseries order\_id & \bfseries ac\_no & \bfseries name &\bfseries city &\bfseries date & \bfseries item &\bfseries quantity &\bfseries item\_price &\bfseries total\_price
		\csvreader[head to column names]{flat_file.csv}{}% use head of csv as column names
		{\\\hline\Order & \AccountNumber &  \Name & \Address & \OrderDate & \ItemName & \ItemQty & \euro\ItemPrice & \euro\OrderTotal}
		\\\hline
	\end{tabular}
\end{center}

\subsubsection{Second normal form}
All the attributes in the table should be \textbf{functionally dependent} on the \emph{primary key}.

$\phi(a)\phi(b) = \phi (ab)$
\end{document}