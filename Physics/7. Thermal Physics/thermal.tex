\documentclass[11pt]{article}
\usepackage[
top    = 2.50cm,% presumably you don't want it to be 0pt as well?
bottom = 2.50cm,
left   = 2cm,
right  = 2cm,
marginparsep = 0pt,
marginparwidth=0pt,
]{geometry}

\usepackage{amssymb}
\usepackage{fancyhdr}
\usepackage[most]{tcolorbox}
\usepackage{siunitx}
\usepackage{amsmath}
\usepackage{tikz}
\usetikzlibrary{decorations.markings}
\usepackage{caption}
\usepackage{pgfplots}
\usepackage{multicol}
\pagestyle{fancy}
\fancyhead[l]{Heat - Abridged edition}
\fancyhead[r]{Giorgio Grigolo {\textcopyright}}
\tikzset{
	mypath/.style={
		postaction=decorate,
		decoration={markings,
			mark=at position #1 with {\coordinate (x);\arrow{>}}},
		thick},
	>=stealth
}
\begin{document}                          
\section{First Law of Thermodynamics}
	\begin{center}
		\textit{The net change in the internal energy of a system is the sum of the net heat transfer from or to the system and the work done by or on the system.}
\end{center}
\begin{equation}
	\Delta U = \Delta Q + \Delta W\tag{\si\joule}
\end{equation}
\begin{center}
	Where $\Delta U$ is the net change in internal energy, $\Delta Q$ is the net amuont of heat transferred and $\Delta W$ is the net amount of work done.
\end{center}
\section{Specific Heat Capacity: }

\begin{equation}
	\Delta Q = mc\Delta\theta\tag{\si{\joule}}
\end{equation}

\section{Work done during volume changes: }
\begin{equation}
	\Delta W = p \Delta V\tag{\si{\joule}}
\end{equation}
\begin{center}
	Where $p$ is the \textbf{pressure} (\si{\pascal}) exerted and $\Delta V$ is the change in \textbf{volume} (\si{\meter\cubed}). 
\end{center}
\subsection{Graph: }
\begin{center}
	\begin{tikzpicture}[scale=1.8]
	\draw[<->] (4.5,0) node[below] {$V$} -| (0,3.5) node[left] {$p$};
	\path (1,1) coordinate (2) {} (2.5,.8) coordinate (1) {}
	(1,2) coordinate (3) {} (2.5,1.5) coordinate (4) {};
	\path[font=\footnotesize] (1) node[below right] {1}
	(2) node[left] {2}
	(3) node[above right] {3}
	(4) node[above right] {4};
	\draw[mypath=.4,shorten <=-.15cm] (1) to[bend left=5] (2);
	\path (x) node[below] {$\mathrm{d}Q=0$};
	\draw[mypath=.4,shorten <=-.2cm,shorten >=-.2cm] (3) to[bend right=15] (4);
	\path (x) node[above right] {$\mathrm{d}Q=0$};
	\draw[mypath=.5] (2) -- (3);
	\draw[->] (x)++(-.5,0) -- + (.3,0) node[midway,above] {$\Delta Q_h$};
	\draw[mypath=.5] (4) -- (1);
	\draw[<-] (x)++(.5,0) -- + (-.3,0) node[midway,above] {$\Delta Q_c$};
	\draw[dashed] (2) -- (1,0) node[below] {$V_2$}
	(1) -- (2.5,0) node[below] {$V_1$};
	\foreach \i in {1,...,4} \filldraw[fill=white] (\i) circle (1pt);
\end{tikzpicture}
\end{center}

\section{Latent heat: }
\begin{center}
	\textit{The heat required to convert a solid into a liquid or vapour, or a liquid into a vapour, without change of temperature.}
\end{center}

\begin{equation}
\Delta Q =ml\tag{\si{\joule\per\kg}}
\end{equation}

\begin{center}
	
\end{center}
\end{document}