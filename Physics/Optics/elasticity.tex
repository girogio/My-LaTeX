\documentclass[11pt]{article}
\usepackage[
top    = 2.50cm,% presumably you don't want it to be 0pt as well?
bottom = 2.50cm,
left   = 2cm,
right  = 2cm,
marginparsep = 0pt,
marginparwidth=0pt,
]{geometry}

\usepackage{amssymb}
\usepackage{fancyhdr}
\usepackage{siunitx}
\usepackage{caption}
\usepackage{graphics}
\usepackage{tikz}
\usepackage{pgfplots}                
\usetikzlibrary{intersections}
\usetikzlibrary{calc}
\usetikzlibrary{arrows.meta}
\usetikzlibrary{shapes.misc}
\usepackage{multicol}
\usepackage{amsmath}
\pagestyle{fancy}
\fancyhead[l]{Elasticity - Abridged edition}
\fancyhead[r]{Giorgio G.}
\fancyfoot[c]{ }
\fancyfoot[r]{\thepage}
\tikzset{
	crossp/.style={
		thick,
		draw=gray,
		cross out,
		inner sep=0pt,
		outer sep=0pt,
		minimum size=2*(#1-\pgflinewidth),
	},
}
\pgfplotsset{compat=newest}

\begin{document}
	\makeatletter
	\newcommand\transformxdimension[1]{
		\pgfmathparse{((#1/\pgfplots@x@veclength)+\pgfplots@data@scale@trafo@SHIFT@x)/10^\pgfplots@data@scale@trafo@EXPONENT@x}
	}
	
	\newcommand\transformydimension[1]{
		\pgfmathparse{((#1/\pgfplots@y@veclength)+\pgfplots@data@scale@trafo@SHIFT@y)/10^\pgfplots@data@scale@trafo@EXPONENT@y}
	}
	\makeatother
	
	\newcommand*{\ShowIntersection}{%
		\fill
		[
		name intersections={of=Hardening and Hooke, name=i, total=\t},
		fill=red
		]
		\foreach \s in {1,...,\t}{(i-\s) circle (2pt)
			node [above left] {\s} (i-\s) node [below right] {%
				\pgfgetlastxy{\macrox}{\macroy}
				\transformxdimension{\macrox}
				\pgfmathprintnumber{\pgfmathresult},
				\transformydimension{\macroy}
				\pgfmathprintnumber{\pgfmathresult}}};}
	
	\section{Hooke's Law: } 
	
	\begin{equation}
		F = k \Delta L\tag{\si{\newton}}
	\end{equation}
	
	\begin{center}
		Where k is the \textbf{stiffness constant} ($\si{\newton\per\metre}$). Valid until the elastic limit.
	\end{center}
	
	\section{Stress: }
	\begin{equation}
		\sigma = \frac{F}{A}\tag{\si{\pascal}}
	\end{equation}
	\begin{center}
		Where $F$ is the force exerted and $A$ is the area on which the force has been exerted.
	\end{center}
	
	\section{Strain: }
	\begin{equation}
		\varepsilon = \frac{\Delta l}{l}\tag{Dimension-less}
	\end{equation}
	\begin{center}
		Where $\Delta l$ is the change in length, and $l$ is the original length.
	\end{center}
	
	\section{Young's Modulus:}
	\begin{equation}
		Y = \frac{\sigma}{\varepsilon}\tag{Dimension-less}
	\end{equation}
	\begin{center}
		We can define Young's modulus to be the measure of stiffness of a given material\footnote{It has to be considered seperately for objects made of different materials.}. 
	\end{center}
	\section{Stress vs Strain graph: }
	\begin{center}
		\begin{tikzpicture}[scale=0.7]
			\begin{axis}[
				x={(2cm,0)},
				y={(0,0.02cm)},
				axis y line=left,
				axis x line=left,
				axis line style=
				{-{Stealth[inset=1pt, angle=30:15pt]}, very thick},
				ymin=0,     % start the diagram at this y-coordinate
				ymax=500,   % end   the diagram at this y-coordinate
				xmin = 0,
				xmax = 7,
				ylabel style={rotate=-90},
				every axis y label/.style=
				{at={(ticklabel* cs:1.02)},anchor=south,},
				ylabel=Stress / $\si{\pascal}$,
				every axis x label/.style=
				{at={(ticklabel* cs:1.02)},below left = 8pt},
				every tick/.style={thick},
				ytick={0,100,...,400},
				xtick={0,1,...,6},
				yticklabels={0,100,200,300,400},
				xlabel=Strain,
				xticklabels={0,1,...,6},
				minor y tick num={1},
				minor x tick num={4},
				tick align=outside]
				
				\addplot[domain=0:1]{300*x};
				
				\coordinate (A) at (1,300);
				\coordinate (B) at (4,450);
				\coordinate (C) at (6,400);
				\coordinate (P) at (0.2,0);
				
				
				
				\node[crossp=5pt, rotate=130, label=-130:{Break}] at (C) {};
				
				\draw[name path global=Hardening]
				(A) .. controls +(71.5651:1.637cm) and +(180:2cm) .. (B);
				
				
				
				\draw[black] (B) .. controls +(0:2cm) and +(130:5mm) .. (C);
			\end{axis}
		\end{tikzpicture}~\\
		\textbf{Gradient: Young's modulus}	
	\end{center}
	
	\section{Force vs Extension graph: }
	\begin{center}
		\begin{tikzpicture}[scale=0.7]
			\begin{axis}[
				x={(2cm,0)},
				y={(0,0.02cm)},
				axis y line=left,
				axis x line=left,
				axis line style=
				{-{Stealth[inset=1pt, angle=30:15pt]}, very thick},
				ymin=0,     % start the diagram at this y-coordinate
				ymax=500,   % end   the diagram at this y-coordinate
				xmin = 0,
				xmax = 5,
				ylabel style={rotate=-90},
				every axis y label/.style=
				{at={(ticklabel* cs:1.02)},anchor=south,},
				ylabel=Force / \si{\newton},
				every axis x label/.style=
				{at={(ticklabel* cs:1.02)},below left = 8pt},
				every tick/.style={thick},
				ytick={0,100,...,400},
				xtick={0,1,2,...,4},
				yticklabels={0,100,200,300,400},
				xlabel=$\Delta l$ / \si{\meter},
				xticklabels={0,1,...,6},
				minor y tick num={1},
				minor x tick num={4},
				tick align=outside]
				
				\addplot[domain=0:4]{100*x};
				
			\end{axis}
		\end{tikzpicture}~\\
		\textbf{Gradient: Energy stored (\si{\joule})}	
	\end{center}
	
	\section{Energy stored \textit{per unit volume}: }
	
	\begin{equation}
		\frac{1}{2}\quad \text{stress} \times \text{strain} \tag{\si{\joule\per\metre\cubed}}
	\end{equation}
	
	\section{Nomenclature: }
	
	\begin{center}
		\resizebox{175pt}{!}{%
		\begin{tabular}{|c|c|}
			\hline
		&	\\Term       & Description                                                       \\&\\ \hline
		&	\\Stiffness  & How difficult to deform                                           \\&\\ \hline
	&	\\Strength   & How difficult to break                                            \\&\\ \hline
	&	\\	Elasticity & Where Young's modulus works.                                      \\&\\ \hline
	&	\\	Ductile    & \begin{tabular}[c]{@{}c@{}}Able to undergo a lot of               \\ plastic behaviour.\end{tabular}                              \\&\\ \hline
	&	\\	Brittle    & \begin{tabular}[c]{@{}c@{}}Breaks suddenly without                \\ undergoing plastic behaviour\end{tabular}                    \\&\\ \hline
	&	\\	Necking    & \begin{tabular}[c]{@{}c@{}}Narrowing of material due              \\ to stretching.\end{tabular}                                \\&\\ \hline
	&	\\	UTS        & \begin{tabular}[c]{@{}c@{}}Ultimate tensile stress a material can \\ withstand w/o fractures/narrowing\end{tabular} \\&\\ \hline
		\end{tabular}}
	\end{center}
	
\end{document}     