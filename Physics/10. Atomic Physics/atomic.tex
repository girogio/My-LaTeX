\documentclass[11pt]{article}
\usepackage[
top    = 2.50cm,% presumably you don't want it to be 0pt as well?
bottom = 2.50cm,
left   = 2cm,
right  = 2cm,
marginparsep = 0pt,
marginparwidth=0pt,
]{geometry}

\usepackage{amssymb}
\usepackage{fancyhdr}
\usepackage{siunitx}
\usepackage{caption}
\usepackage{graphics}
\usepackage{tikz}
\usepackage{pgfplots}                
\usetikzlibrary{intersections}
\usetikzlibrary{calc}
\usetikzlibrary{arrows.meta}
\usetikzlibrary{shapes.misc}
\usepackage{multicol}
\usepackage{amsmath}
\pagestyle{fancy}
\fancyhead[l]{Atomic Physics - Abridged edition}
\fancyhead[r]{Giorgio G.}
\fancyfoot[c]{ }
\fancyfoot[r]{\thepage}
\tikzset{
	crossp/.style={
		thick,
		draw=gray,
		cross out,
		inner sep=0pt,
		outer sep=0pt,
		minimum size=2*(#1-\pgflinewidth),
	},
}
\pgfplotsset{compat=newest}

\begin{document}
	\section{Electron-volt to Joule conversion: }
	\begin{equation}
		1\si{\electronvolt} = 1.6\times10^{-19}\si{\joule}\notag
	\end{equation}
\section{Energy level equation: }
\begin{equation}
	\Delta E = hf\quad;\quad f=\frac{c}{\lambda}\tag{\si{\joule} ; \si{\hertz}}
\end{equation}
\begin{center}
	Where $\Delta E$ is the difference between two energy levels (\si{\joule}), $h$ is Planck's constant ($6.63\times10^{-34}\,\si{\meter\squared\kilogram\per\second}$), $c$ is the speed of light ($3\times10^8\,\si{\meter\per\second}$) and $\lambda$ is the wavelength (\si{\meter}) of the particle.
\end{center}

\section{Nomenclature: }
\begin{center}
		\begin{tabular}{|c|c|}
			\hline
			&	\\Term       & Description                                                       \\&\\ \hline
			&	\\Excited  & When the atom goes $\uparrow$ it is less stable\\&\\ \hline
			&	\\Grounded   & The state in which the atom is most relaxed, located $\downarrow$\\&\\ \hline
		\end{tabular}
\end{center}
\end{document}     