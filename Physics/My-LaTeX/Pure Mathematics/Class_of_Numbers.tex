\documentclass{standalone}
\begin{document}
	\chapter{Classification of numbers}
	
	\begin{itemize}
		\item{Natural numbers: $\mathbb{N}$; (1, 2, 3, 4 \ldots)}
		\subitem{This set includes every number which is both positive and whole.}
		\item{Integer numbers: $\mathbb{Z}$; (-2, -1, 0, 1, 2, \ldots)}
		\subitem{The integer number set includes every negative and positive whole numbers, similarly to $\mathbb{N}$}
		\item{Rational numbers: $\mathbb{Q}$; (-1, 2 , $\frac{1}{2}$)}
		\subitem{A number is s.t.b rational if expressed in the form $\frac{p}{q};p, q \in \mathbb{Z}$.}
		\item{Irrational numbers: $\mathbb{Q'}$; ($\pi, e, \sqrt{2}, \sqrt{5}, \ldots$))}
		\subitem{If a number is not classified as any of the above, it is referred to as irrational.}
		\item{Real numbers: $\mathbb{R}$}
		\subitem{Anything mentioned above inclusively represent the set of Real numbers}\\
	\end{itemize}
	
	
	\emph{We can additionally refer to positive or negative numbers in any set by using the notation:}\\
	\begin{center}
		$\mathbb{R}^+$ and $\mathbb{R}^-$
	\end{center}
	%TODO: insert image of sets
	
	
	\newpage	
	\end{document}