\documentclass{standalone}


\begin{document}
	
	\chapter{Series}
	
	\section{Sequences}
	
	Consider the following set of numbers: 
	\begin{center}
		$2,4,6,8,...$\\
		$1,2,4,8,16,...$\\
		$4,9,16,25,...$\\
	\end{center}
	In each of the above cases, the numbers are written in a particular order and there is a clear rule for obtaining the next number and as many numbers in the list.\\
	
	The above are all examples of sequences where a sequence is a set of terms in a defined order with a rule for obtaining the terms.
	
	\section{Summation Series}
	
	When the terms of a sequence are added, a summation series is formed:
	
	\begin{multicols}{2}
		\begin{center}
			$2+4+6+8+10+...$\\
			$1+2+4+8+16+...$\\
			$4+9+16+25+36+...$\\
		\end{center}
		\begin{center}
			~\\
			are all examples of series.
			~\\
		\end{center}
	\end{multicols}
	A series can be finite or infinite, where a finite series consists of a fixed number of terms, whereas an infinite series has an infinite number of terms.\\
	
	Considering the follwing series, \[1+\frac{1}{2}+ \frac{1}{4}+\frac{1}{8}+\frac{1}{16}+...\] we can notice that the general term of this series is $\frac{1}{2^r}$. The general term of a series is not unique, it depends on the initial value of $r$. Thus, the general term $\frac{1}{2^{r-1}}$ also corresponds to the above series but we take the initial value of $r$ to be 1.\\
	
	A summation series can be defined in a concise way using the Greek letter $\Sigma$ denoting the \textit{summation of terms}. The above series may be expressed as \[\sum_{r=0}^\infty \frac{1}{2^r}\] which is equivalent to \[\sum_{r=1}^\infty \frac{1}{2^{r-1}} \quad r \in \mathbb{Z}\]
	
	\section{Arithmetic Progressions}
	\subsection{Definition}
	An arithmetic progression is a sequence of numbers starting with term $a$, in which successive terms are obtained by adding the same constant, denoted by $d$, reffered to as the \textbf{common difference}.\\
	
	\subsection{General term}
	Let us consider the general A.P with first term $a$ and common difference $d\colon$ \[a+(a+d)+(a+2d)+(a+3d)+(a+4d)+(a+5d)+...\] By observing the coefficient of $d$ and the position of the term, we can conclude that the general term can be obtained by the equation: \[T_n = a+(n-1)d\]
	\subsection{Sum of the first n terms}
	Considering an A.P. with $n$ terms, let the first term be $a$, the common difference to be $d$	and the last term to be $l$
	
	\begin{align*}
				S_n &= a+(a+d)+(a+2d)+(a+3d)+(a+4d)+(a+5d)+...+(l-d)+l\\
\implies	S_n &= l+(l-d)+(l-2d)+(l-3d)+(l-4d)+(l-5d)+...+(a+d)+a\\
	\implies	2S_n &= n(2a+l)\\
 \implies   S_n  &= \frac n2 (2a+l)\\
	   		&= \frac n2 (2a +(n-1)d)
	\end{align*}
	\newpage
	\section{Geometric progressions}
	\subsection{Definition}
	A geometric progression is a sequence of numbers starting with term $a$, in which successive terms are obtained by multiplying the same constant, denoted by $r$, reffered to as the \textbf{common ratio}.\\
	
	\subsection{General term}
	
	Let us consider a general G.P. with first tirm $a$ and a common ratio $r$: \[a+ar+ar^2+ar^3+ar^4+ar^5+...+ar^n-1+\ldots\]
	By observing the exponent of $r$ and the position of the term we can conclude that the general term can be obtained by the equation: \[\mathrm{T_n} = ar^{n-1}\]
	
	\subsection{Sum of the first n terms}
	Considering a G.P. with first term $a$, common ratio $d$ and it's sum denoted by $S_n$:
	
	\begin{align*}
		S_n = a+&ar+ar^2+ar^3+ar^4+ar^5+\ldots+ar^{n-1}+\ldots\\
		rS_n = \phantom{a+}&ar+ar^2+ar^3+ar^4+ar^5+\ldots+ar^n+\ldots\\
		S_n - rS_n =\phantom{a+}& a-ar^n\\
		(1-r)S_n = \phantom{a+}&a(1-r^n)\\
		S_n = \phantom{a+}&  \frac{a(1-r^n)}{1-r} \qed
		\end{align*}
	
	
	\section{Convergence of a series}
	
	For any\footnote{A.P's by definition \textbf{do not} converge.} given G.P., we can always add infinitely many times more terms to the end of the series, but what hapens to it's sum? Given that $\abs r = 1$, the sum will always get infinitesimally closer to a definite number.\\
	
	Considering the G.P. $\frac 12 + \frac 14 + \frac 18 + \frac 1 {16} + \ldots$ we can observe that if you keep on adding terms and summing them, the sum will approach 1. In mathematical language we say \[\text{as } n \rightarrow \infty\,,\,S_n = 1\] or \[\lim_{n\to\infty}S_n = 1\]
	
	\section{Binomial theorem}
	
	The binomial theorem states that \[(x+y)^n = \sum_{k=0}^n\begin{pmatrix}n\\k\end{pmatrix} x^ky^{n-k}	\]
	
	\section{Maclaurin Series}
	\subsection{Derivation}
	Let $f(x)$ be nay function of $ x $ and suppose that $ f(x) $ can be expanded as a series of ascending powers of $ x $ and that this series can be differentiated $ w.r.t. x $
	\begin{center}
		$	f(x) \equiv a_0   + a_{1}x  + a_{2}x^2 + a_{3}x^3 + a_{4}x^4 + ... +	 a_{r}x^r$\\
		\text{where $a_n$ are constants to be found}
	\end{center}
	Thus, inputting $0$ into $f(x)$ returns:
	$$\boxed{f(0) = a_{0}}$$\\
	Differentiating  $f(x)$ $w.r.t.x\colon$ 
	$$f'(x) \equiv a_{1} + 2a_{2}x + 3a_{3}x^2 + 4a_4x^3 + \cdots + ra_{r}x^{r-1} + \cdots$$
	Inputting $0$ into $f'(x)\colon$
	$$\boxed{f'(0) = a_1}$$\\
	Differentiating $f'(x)$ $w.r.t.x\colon$
	$$f''(x) \equiv 2a_{2} + 6a_{3}x + 12a_4x^2 + \cdots + (r-1)(r)a_{r}x^{r-2} + \cdots$$
	Inputting $0$ into $f''(x)\colon$
	$$\boxed{f''(0) = 2a_2}$$\\
	Differentiating $f''(x)$ $w.r.t.x\colon$
	$$f'''(x) \equiv 6a_{3} +24a_4x + \cdots + (r-2)(r-1)(r)a_{r}x^{r-3}+ \cdots$$
	Inputting $0$ into $f'''(x)\colon$
	$$\boxed{f'''(0) = (2)(3)a_3}$$
	$$\vdots$$
	\newpage
	By the above calculation we can conclude that: 
	$$\boxed{a_r = \frac{f^r(x)}{r!}}$$
	Considering all of the above: 
	
	\tcbset{
		enhanced,
		colback=red!5!white,
		boxrule=0.1pt,
		colframe=black!75!black,
		fonttitle=\bfseries
	}
	\begin{center}
		\begin{tcolorbox}[center title,hbox,    %%<<---- here
			lifted shadow={1mm}{-2mm}{3mm}{0.1mm}%
			{black!50!white}]
			\begin{varwidth}{\textwidth}
				\begin{center}
					$f(x) \equiv f(0) + f'(0)x + \dfrac{f''(0)x^2}{2!} + \dfrac{f'''(0)x^3}{3!} + \cdots + \dfrac{f^r(0)x^r}{r!} + \cdots$\\
					\bigskip
					$\displaystyle \therefore \quad f(x) \equiv \sum_{r=1}^{\infty} \dfrac{f^r(x)}{r!}$
				\end{center}
			\end{varwidth}
		\end{tcolorbox} 
	\end{center}
	
	
	This is known as Maclaurin's Theorem, and can be obtained if and only if $f^r(0) \in \mathbb{R}$. In the following examples we use Maclaurin's Theorem to obtain the series expansion of some standard equations. The range of validity of each expansion is left as an exercise to the reader.
	\subsection{Examples}
	\begin{example}
		\emph{Express} $e^x$ \emph{as a series expansion using the Maclaurin theorem.}
	\end{example}
	Let $f(x) = e^x$
	\begin{alignat*}{5}
		&   & f(x)    & = & e^x \quad & \Rightarrow\quad & f(0) = 1    \\
		&   & f'(x)   & = & e^x \quad & \Rightarrow\quad & f'(0) = 1   \\
		&   & f''(x)  & = & e^x \quad & \Rightarrow\quad & f''(0) = 1  \\
		&   & f'''(x) & = & e^x \quad & \Rightarrow\quad & f'''(0) = 1 
	\end{alignat*}
	
	$$\therefore \quad e^x  = 1 + x + \frac{x^2}{2!} + \frac{x^3}{3!} + \frac{x^4}{4!} +\ldots + \frac{x^r}{r!} + \ldots $$
	\hrulefill
	\begin{example}
		\emph{Express} $\cos x$ \emph{as a series expansion using the Maclaurin theorem.}
	\end{example}
	\begin{alignat*}{5}
		&   & f(x)    & =           & \cos (x) \quad & \Rightarrow\quad & f(0) = 1    \\
		&   & f'(x)   & = \text{ -} & \sin (x) \quad & \Rightarrow\quad & f'(0) = 1   \\
		&   & f''(x)  & = \text{ -} & \cos(x) \quad  & \Rightarrow\quad & f''(0) = 1  \\
		&   & f'''(x) & =           & \sin (x) \quad & \Rightarrow\quad & f'''(0) = 1 
	\end{alignat*}
	$$\therefore \quad \cos (x)  = 1 - \frac{x^2}{2!} + \frac{x^4}{4!} -\frac{x^6}{6!}+\ldots + (-1)^r\times\frac{x^{2r}}{2r!} + \ldots $$\\
	The above expansion justifies the fact that when $x$ is very small and thus high powers of $x$ may be neglected, then: \boxed{\cos x \approx 1- \frac{x^2}{2}}\\
	\begin{example}
		\emph{Express} $\ln(1+x)$ \emph{as a series expansion using the Maclaurin theorem.}
	\end{example}
	
	\begin{alignat*}{5}
		&   & f(x)    & = & \ln(1+x)      \quad & \Rightarrow  \quad & f(0)    & = 0  \\
		&   & f'(x)   & = & (x+1)^{-1} \quad    & \Rightarrow  \quad & f'(0)   & = 1  \\
		&   & f''(x)  & = & -(1+x)^{-2}  \quad  & \Rightarrow \quad  & f''(0)  & = -1 \\
		&   & f'''(x) & = & 2(1+x)^{-3} \quad   & \Rightarrow  \quad & f'''(0) & = 2  
	\end{alignat*}
	$$\therefore \quad \ln(1+x)  = x - \frac{x^2}{2} + \frac{x^3}{3} - \ldots + (-1)^{r+1}\times\frac{x^r}{r} + \ldots $$\\
	\hrulefill
	
	\begin{example}
		\emph{Expand} $\arcsin(x)$ \emph{up to the term in $x^3$. By putting $x=\frac{1}{2}$, find an approximate value for $\pi$}
	\end{example}
	
	\begin{alignat*}{5}
		& f(x)    & = & \arcsin(x)      \quad                                   & \Rightarrow  \quad & f(0)    & = 0 \\
		& f'(x)   & = & (1-x^2)^{\frac{-1}{2}} \quad                            & \Rightarrow  \quad & f'(0)   & = 1 \\
		& f''(x)  & = & x(1-x^2)^{\frac{-3}{2}}\quad                            & \Rightarrow \quad  & f''(0)  & = 0 \\
		& f'''(x) & = & 3x(1-x^2)^{\frac{-5}{2}} + (1-x^2)^{\frac{-3}{2}} \quad & \Rightarrow  \quad & f'''(0) & = 1 
	\end{alignat*}
	$$\therefore \quad \arcsin(x)   = x + \frac{x^3}{3!} + \ldots $$\\
	Putting $x = \frac{1}{2}$\\
	\begin{alignat*}{2}
		&          & f\left(\frac{1}{2}\right) & = \frac{\pi}{6}                                \\
		& \implies & \pi                       & \approx 6\left(\frac{1}{2}+\frac{1}{81}\right) \\
		& \implies & \pi                       & \approx \frac{83}{27}                          
	\end{alignat*}
	\newpage
	\subsection{Expanding compound functions using standard functions}
	\begin{example}
		Expand $a)\quad \dfrac{e^{2x}+ e^{-3x}}{e^x}\quad b)\quad \ln\left( \dfrac{1-2x}{(1+2x)^2}\right)  $as series of ascending powers of $x$ up to the term in $x^4$. Give the general term in each case and the range of values of $x$ for which each expansion is valid.
	\end{example}
	\begin{alignat*}{2}
		a) &                  & e^x                                    & = 1 + x + \frac{x^2}{2!} + \frac{x^3}{3!} + \frac{x^4}{4!} +\ldots                                                                                                  \\
		&                  & e^{-3x}                                & = 1 + (-3)x + \frac{(-3x)^2}{2!} + \frac{(-3x)^3}{3!} + \frac{(-3x)^4}{4!} +\ldots                                                                                  \\
		& \therefore \quad & e^{x} + e^{-3x}                        & = \left (1 + x + \frac{x^2}{2!} + \frac{x^3}{3!} + \frac{x^4}{4!}\right ) + \left (1 + (-3)x + \frac{(-3x)^2}{2!} + \frac{(-3x)^3}{3!} + \frac{(-3x)^4}{4!}\right ) \\
		&                  &                                        & = 2 -2x + \frac{10x^2}{2!} - \frac{26x^3}{3!} + \frac{89x^4}{4!}                                                                                                    \\
		\bigskip\\
		b) &                  & \ln\left(\dfrac{1-2x}{(1+2x)^2}\right) & = \ln(1-2x) -2\left (\ln(1+2x)\right)                                                                                                                               
		\intertext{\emph{Consider $\ln(1-2x)\colon$}}
		&                  & \ln\left(1+(-2x)\right)                & = -2x + -2x^2-\frac{8x^3}{3}-4x^4 + \ldots + \frac{(-1)^{r-1}(-2x)^r}{r}+ \ldots                                                                                    \\
		\intertext{\emph{Consider $\ln(1+2x)\colon$}}
		&                  & \ln\left(1+2x\right)                   & = 2x + -2x^2+\frac{8x^3}{3}-4x^4 +\ldots + \frac{-2(-1)^{r-1}(2x)^r}{r} + \ldots                                                                                    \\ 
		& \therefore       & \ln\left(\dfrac{1-2x}{(1+2x)^2}\right) & = \left( -2x + -2x^2-\frac{8x^3}{3}-4x^4\right) -2\left(2x + -2x^2+\frac{8x^3}{3}-4x^4\right)                                                                       \\
		&                  &                                        & = -6x+2x^2-8x^3	+4x^{4}                                                                                                                                             \\
	\end{alignat*}
	\text{Range of Validity:}
	\begin{multicols}{2}
		\begin{center}
			$$	\quad\frac{(-1)^{r-1}(-2x)^r}{r} - \frac{2(-1)^{r-1}(2x)^r}{r}$$\\
			$$  =\frac{(-1)^{r-1}(-1)^r(2x)^r+2(-1)^r(2x)^r}{r}$$\\
			$$	=\frac{(-1)^{2r-1}(2x)+2(-1)^r(2x)^r}{r}$$\\
			$$	=\frac{\left( (-1)^{2r-1}+2(-1)^r\right) (2x)^r}{r}$$\\
			$$  =\frac{ \left(-1+2(-1)^r\right(2x)^r)}{r}$$
			$$ =\frac{2^r(2(-1)^r -1)x^r}{r}$$							
		\end{center}
	\end{multicols}
	
	\begin{example}
		Expand  $\ln\left(\frac{x+1}{x}\right)\quad$as series of ascending powers of $x$ up to the term in $x^4$. Give the general term in each case and the range of values of $x$ for which each expansion is valid.
	\end{example}
	
	$$f(x) \ln\left(\frac{x+1}{x}\right)  = \ln\left(1+\frac{1}{x}\right ) $$
	
	\begin{alignat*}{5}
		&   & f(x)    & = & \ln(1+\frac{1}{x})    \quad & \Rightarrow  \quad & f(0)    & = 0  \\
		&   & f'(x)   & = & (x+1)^{-1} \quad            & \Rightarrow  \quad & f'(0)   & = 1  \\
		&   & f''(x)  & = & -(1+x)^{-2}  \quad          & \Rightarrow \quad  & f''(0)  & = -1 \\
		&   & f'''(x) & = & 2(1+x)^{-3} \quad           & \Rightarrow  \quad & f'''(0) & = 2  
	\end{alignat*}
	
	\begin{center}
		$=\frac{1}{x} - \frac{1}{2x^2} + \frac{1}{3x^3} - \frac{1}{4x^4} + \ldots + \frac{(-1)^{r+1}}{r} $
	\end{center}
	\hrulefill
	\begin{example}
		Expand $\sin^2{x}$ using Maclaurin's series up to $x^4$
	\end{example}
	\bigskip
	$$sin^2(x)\equiv\frac{1-\cos(2x)}{2}$$
	\begin{alignat*}{2}
		\intertext{\emph{Consider} $\cos(2x)\colon$} &                 &           & =1-\frac{(2x)^{2}}{2!} + \frac{(2x)^4}{4!}  - \cdots + \frac{(-1)^r(2x)^{2r}}{(2r)!}+\cdots           \\
		&                 &           & =1-2x^2+\frac{2x^4}{3}-\cdots+\frac{(-1)^r(2x)^{2r}}{(2r)!}+\cdots                                    \\
		& \therefore\quad & \sin^2(x) & \equiv \frac{1}{2}\left( 1-(1-2x^2+\frac{2x^4}{3}-\cdots+\frac{(-1)^r(2x)^{2r}}{(2r)!}+\cdots)\right) \\
		&                 &           & =\frac{1}{2}\left(1-1+2x^2 - \frac{2x^4}{3} +\cdots+\frac{(-1)^r(2x)^{2r}}{(2r)!}+\cdots\right)       \\
		&                 &           & =x^2-\frac{x^4}{3} +\cdots+\frac{(-1)^{r+1}(2x)^{2r}}{(2r)!}+\cdots                                   
	\end{alignat*}
	\newpage
	\begin{example}
		Given $e^{2x}\cdot \ln{1+ax}$ find possible values for $p$ and $q$.
	\end{example}
	\begin{alignat*}{2}
		\intertext{\emph{Consider} $e^{2x}\colon$}
		&                 & e^{2x}                & = 1+2x + \frac{(2x)^2}{2!} + \frac{(2x)^{3}}{3!} +\cdots +  \frac{x^r}{r!} + \cdots           \\
		\intertext{\emph{Consider} $\ln(1+ax)\colon$}
		&                 & \ln(1+ax)             & = ax - \frac{(ax)^2}{2} + \frac{(ax)^3}{3} - \cdots + \frac{(-1)^{r+1}x^r}{r} + \cdots        \\
		& \therefore\quad & e^{2x}\cdot \ln(1+ax) & = \left(1+2x+2x^2+\frac{4x^3}{3}\right) \left(ax - \frac{a^2x^2}{2} + \frac{a^3x^3}{3}\right) \\
		&                 &                       & =ax-\frac{a^2x^2}{2}+\frac{a^3x^3}{3}+2ax^2-2a^2x^3                                           \\
		&                 &                       & =ax-\left(\frac{a^2}{2}+2a\right) x^2+\left( \frac{a^3}{3}-2a^2\right) x^3                    
	\end{alignat*}
	~\\
	\begin{equation*}
		\therefore\left.
		\begin{alignedat}{2}
			\hspace{0.5 in}	&&p&=a\\
			&&	\frac{a^2}{2}+2a&=\frac{-3}{2}\\
			&&	\frac{a^3}{3}-2a^2&=q
		\end{alignedat}	\right\}
	\end{equation*}
	
	\begin{center}
		$$ p = -3,-1 $$
		$$ q= -27, -\dfrac{7}{3}$$
	\end{center}
	\hrulefill
	\newpage
	\section{Summation of Series}
	\subsection{Method 1: Generating differences}
	
	\begin{example}
		Simplify $f(r)-f(r+1)$, when $f(x) = \frac{1}{r^2}$. Hence, find the sum up to $n$ terms of:\\ $$\sigma_1 = \frac{3}{1^2\cdot 2^2} + \frac{5}{2^2\cdot 3^2} + \frac{7}{3^2\cdot 4^2} + \ldots$$
	\end{example}
	~\\
	Simplifying $f(r) - f(r+1)\colon$
	\begin{alignat*}{2}
		&   & f(r)-f(r+1) & = \frac{1}{r^{2}} - \frac{1}{(r+1)^{2}} \\
		&   &             & =\frac{(r+1)^{2}-r^2}{r^2(r+1)^{2}}     \\
		&   &             & =\frac{2r+1}{r^2(r+1)^2}                
	\end{alignat*}
	\textit{Generating series and adding}\\
	\textit{quantitatively equivalent terms:}
	\begin{center}
		$	\frac{1}{1^2} - \cancel{\frac{1}{2^2}}$
		$$	\cancel{\frac{1}{2^2}} - \cancel{\frac{1}{3^2}}$$
		$$	\cancel{\frac{1}{3^2}} - \cancel{\frac{1}{4^2}}$$
		$$\vdots$$
		$$\cancel{\frac{1}{n^2}} - \frac{1}{n+1^2}$$
		~\\
		$$\boxed{\therefore \quad \sigma_1 = 1 - \frac{1}{n+1^2}}$$
	\end{center}
	\hrulefill
	\begin{example}
		If $f(r) = r(r+1)!$ simplify $f(r) - f(r-1)$. Hence sum the series:\\
	\end{example}
	\begin{center}
		$\sigma_1 = 5\cdot 2! + 10\cdot 3! + 17\cdot 4! +  ... + (n^2=1)n!$\\
	\end{center}
	\hrulefill
	\begin{alignat*}{2}
		&   & f(r)-f(r-1) & =r(r+1)! - (r-1)r!  \\
		&   &             & =r(r+1)r! - (r-1)r! \\
		&   &             & =r!(r^2+r-r+1)      \\
		&   &             & =r!(r^2+1)          
	\end{alignat*}
	\newpage
	\textit{Generating series and adding}\\
	\textit{quantitatively equivalent terms:}
	\begin{center}
		$\bcancel{f(2)} - f(1)$
		$$\bcancel{f(3)} - \bcancel{f(2)}$$
		$$\bcancel{f(4}) - \bcancel{f(3)}	$$
		$$\vdots$$
		$$\bcancel{f(n-1)} - \bcancel{f(n-2)}$$
		$$f(n) - \bcancel{f(n-1)}$$\\
	\end{center}
	\begin{example}
		If $f(r) = \cos2r\theta$, simplify $f(r) - f(r+1)$. Hence find $\sin3\theta + \sin5\theta + \sin7\theta$
		\hrulefill
	\end{example}
	\begin{alignat*}{2}
		&   & f(r)-f(r+1) & =\cos(2r\theta)- \cos(2(r+1)\theta)                                                                            \\
		&   &             & =-2\sin\left(\frac{2r\theta + (2r+2)\theta}{2}\right)\cdot \sin\left(\frac{ 2r\theta - 2(r+1)\theta}{2}\right) \\
		&   &             & =-2\sin(2r\theta+ \theta)\sin(-\theta)                                                                         \\
		&   &             & =2\sin(\theta[2r+1])\sin\theta                                                                                 
	\end{alignat*}
	\textit{Generating series and adding}\\
	\textit{quantitatively equivalent terms:}\\
	\begin{center}
		
		\begin{tabular}{ccccc}
			$r=1$    &   & $2\sin(3\theta)\sin(\theta)$ & $=$ & $f(1)\cancel{-f(2)}$          \\
			$r=2$    &   & $2\sin(5\theta)\sin(\theta)$ & $=$ & $\cancel{f(2)}\cancel{-f(3)}$ \\
			$r=3$    &   & $2\sin(7\theta)\sin(\theta)$ & $=$ & $\cancel{f(3)}\cancel{-f(4)}$ \\
			$\vdots$ &   & $\vdots$                     & $=$ & $\vdots$                      \\
			$r=n$    &   & $2\sin(2n+1)\sin(\theta)$    & $=$ & $\cancel{f(n)}-f(n+1)$        
		\end{tabular}\\
		~\\
		\begin{alignat*}{2}
			&   & f(1) - f(n+1) & = 2\sin(\theta)\sin(2n+1)                                \\
			&   &               & =\frac{\cos(2\theta)- \cos(2\theta(n+1))}{2\sin(\theta)} \\
			&   &               & =\frac{2\sin(\theta(2r+1))\sin\theta}{2\sin(\theta)}     \\
			&   &               & =\frac{sin((n=1)\theta)\sin(n\theta)}{\sin(\theta)}      
		\end{alignat*}
	\end{center} 
	
	
	\subsection{Method 2: Using partial fractions}
	A special case of the previous method can happen to imply a partial fraction decomposition.
	
	\begin{example}
		Decompose $\frac{1}{r(r+1)}$. Hence find the sum of $$\sigma_1 = \frac{1}{1\cdot 2} + \frac{1}{2\cdot 3} + \frac{1}{3\cdot 4} + \cdots$$
	\end{example}~\\
	\textit{Decomposing:}
	$$\frac{1}{r(r+1)} \equiv \frac{1}{r} - \frac{1}{r-1}$$
	{Generating series and adding}\\
	{quantitatively equivalent terms:}\\
	\begin{center}
		\begin{tabular}{ccccc}
			$r=1$    &   & $\frac{1}{1}$          & $-$ & $\cancel{\frac{1}{2}}$\linebreak \\
			&&&\\
			$r=2$    &   & $\cancel{\frac{1}{2}}$ & $-$ & $\cancel{\frac{1}{3}}$           \\
			&&&\\
			$r=3$    &   & $\cancel{\frac{1}{3}}$ & $-$ & $\cancel{\frac{1}{4}}$           \\
			&&&\\
			$\vdots$ &   & $\cancel{\vdots}$      & $-$ & $\cancel{\vdots}$                \\
			&&&\\
			$r=n$    &   & $\cancel{\frac{1}{n}}$ & $-$ & $\frac{1}{n+1}$                  \\
		\end{tabular}\\
	\end{center}
	\begin{alignat*}{2}	
		& \therefore \quad & \sigma_1 & = 1 - \frac{1}{n+1}                                  \\
		\intertext{Finding convergent value: }
		&                  & 1        & =\lim_{x \to \infty} \left( 1 - \frac{1}{n+1}\right) 
	\end{alignat*}
	\hrulefill
	\newpage
	\begin{example}
		Find $\quad \sum_{r=3}^n \frac{2}{(r-1)(r+1)}$
	\end{example}
	\begin{alignat*}{2}	
		\intertext{Consider $\frac{2}{(r-1)(r+1)}\colon$}
		&   & \frac{2}{(r-1)(r+1)}                                      & \equiv \frac{1}{r-1} - \frac{1}{r+1}                           \\
		&   & \therefore \quad \sum_{r=3}^n \dfrac{2}{(r-1)(r+1)} \quad & \equiv \quad  \sum_{r=3}^n \dfrac{1}{(r-1)} - \dfrac{1}{(r+1)} \\
	\end{alignat*}
	{Generating series and adding}\\
	{quantitatively equivalent terms:}\\
	\subsection{Method 3: Using standard results}
	
	\subsection{Method 4: Comparing to standard results}
	
\end{document}e