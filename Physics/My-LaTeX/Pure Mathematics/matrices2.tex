%!TEX root = maths.tex
\newcommand{\adj}{\text{adj}\,}
\newcommand{\identitymatrix}{\begin{bmatrix*}[c]1&0&0\\0&1&0\\0&0&1\end{bmatrix*}}
\begin{document}
\chapter{Matrices II}
\section{Determinant of a 3x3 Matrix}
The determinant of a 3x3 matrix is calculated by extracting a row of 2x2 determinants from the given 3x3 matrix. These 2x2 determinants are referred to as minors.

\begin{example}
	Find the determinant of the matrix $\mathbf A = \left(\begin{smallmatrix*}[r]
	2 &-1 &4\\
	3 &0 &-3\\
	4 &5 &6
	\end{smallmatrix*}\right)$
\end{example}
\section{[Parentheses about vector product]}
Consider the vectors $\mathbf{a} = x_1\mathbf{i} + y_1\mathbf{j} + z_1\mathbf{k}$ and b $\mathbf{b} = x_2\mathbf{i} + y_2\mathbf{j} + z_2\mathbf{k}$.\\

Since vector product is distributive across addition:\\

\begin{align*}
	\mathbf{a} \times \mathbf{b} & = \left(  x_1\mathbf{i} + y_1\mathbf{j} + z_1\mathbf{k} \right) \times \left( x_2\mathbf{i} + y_2\mathbf{j} + z_2\mathbf{k} \right)\\
	&+\cancel{x_1x_2(\mathbf i \times \mathbf i)} + x_1y_2(\mathbf i \times \mathbf j) + x_1z_2(\mathbf i \times \mathbf k)\\
	&+ y_1x_2(\mathbf j \times \mathbf i) + \cancel{y_1y_2(\mathbf j \times \mathbf j)} + y_1z_2(\mathbf j \times \mathbf k)\\
	&+z_1x_2(\mathbf k \times \mathbf i) + z_1y_2(\mathbf k \times \mathbf j) + \cancel{z_1z_2(\mathbf k \times \mathbf k)}\\
	&=x_1y_2\mathbf k - x_1z_2\mathbf j - y_1x_2 \mathbf k + y_1z_2\mathbf i +z_1x_2\mathbf j -z_1y_2\mathbf i\\
	&=(y_1z_2-z_1y_2)\mathbf i - (x_1z_2 - z_1 x_2)\mathbf j + (x_1y_2-y_1x_2)\mathbf k
\end{align*}

\section{Some properties of determinant}
The value of the determinant is unaltered if all the rows and columns of a given matrix are interchanged. If the above happens, the resulting determinant will be of opposite sign.\\
 If one row/column of a determinant $D$ is multiplied by $\lambda$, the resulting determinant is equal to $\lambda D$

\section{The Inverse of a 3x3 matrix}


\subsection{Matrix of cofactors}
Consider the matrix $\mathbf A = \begin{bmatrix}a&b&c\\d&e&f\\g&h&i\end{bmatrix}$\\

\noindent For each element of any given matrix $\mathbf A$:
	\begin{itemize}
\item{Ignore the values of the current row and column.}
\item{Calculate the determinant of the remaining values.}
\item{Apply alternating signs starting from $+$.}
\item{Input this determinant into a new matrix}
	\end{itemize}1
The result would be the below matrix referred to as \emph{the matrix of co-factors}.

\[
\mathbf C=
\begin{bmatrix*}[r]
	\begin{vmatrix}e&f\\h&i\end{vmatrix}&&-\begin{vmatrix}d&g\\f&i\end{vmatrix}&&\begin{vmatrix}d&e\\g&h\end{vmatrix}\phantom{-}\\\\
	-\begin{vmatrix}e&f\\h&i\end{vmatrix}&&\begin{vmatrix}d&g\\f&i\end{vmatrix}&&-\begin{vmatrix}d&e\\g&h\end{vmatrix}\phantom{-}\\\\
	\begin{vmatrix}e&f\\h&i\end{vmatrix}&&-\begin{vmatrix}d&g\\f&i\end{vmatrix}&&\begin{vmatrix}d&e\\g&h\end{vmatrix}\phantom{-}
\end{bmatrix*}
\]
\subsection{Adjugate matrix}
The next step of finding the inverse matrix of $\mathbf A$ is finding the \emph{adjugate} matrix of $\mathbf A$. This is done by obtaining the \emph{transpose} of the matrix of cofactors, in our case, $\mathbf C$.
\[
\adj A =
\begin{bmatrix*}[r]
\begin{vmatrix}e&f\\h&i\end{vmatrix}&&-\begin{vmatrix}e&f\\h&i\end{vmatrix}&&\begin{vmatrix}e&f\\h&i\end{vmatrix}\phantom{-}\\\\
-\begin{vmatrix}d&g\\f&i\end{vmatrix}&&\begin{vmatrix}d&g\\f&i\end{vmatrix}&&-\begin{vmatrix}d&g\\f&i\end{vmatrix}\phantom{-}\\\\
\begin{vmatrix}d&e\\g&h\end{vmatrix}&&-\begin{vmatrix}d&e\\g&h\end{vmatrix}&&\begin{vmatrix}d&e\\g&h\end{vmatrix}\phantom{-}
\end{bmatrix*}
\]
\subsection{Inverse matrix}
Let the inverse of a matrix $A$ be $A^{-1}$. It is defined as a matrix of the same size of $A$ such that \[AA^{-1} = A^{-1}A = I\]
This same matrix $A^{-1}$ is defined more particularly as \[A^{-1} = \frac{1}{\det A}\quad\text{adj }A\]
\begin{example}
	Find the inverse of the matrix $A = \begin{pmatrix*}
										2&3&1\\
										1&1&1\\
										5&-1&0
										\end{pmatrix*}$
	\end{example}
\begin{example}
Solve using the inverse matrix method the system of equations:\\
$x+y+z=7$\\
$x-y+2z=9$\\
$2x+y-z=1$
\end{example}
\newpage
\section{Transformation Matrices in 3D}
\subsection{Reflection along the xy plane}

\begin{center}
\begin{tikzpicture}[x=0.5cm,y=0.5cm,z=0.3cm,>=stealth]
% The axes
\draw[->] (xyz cs:x=-5) -- (xyz cs:x=5) node[above] {$x$};
\draw[->] (xyz cs:y=-5) -- (xyz cs:y=5) node[right] {$y$};
\draw[->] (xyz cs:z=-5) -- (xyz cs:z=5) node[above] {$z$};


\foreach \coo in {-4,-3,...,4}
{
	\draw (\coo,-1.5pt) -- (\coo,1.5pt);
	\draw (-1.5pt,\coo) -- (1.5pt,\coo);
	\draw (xyz cs:y=-0.15pt,z=\coo) -- (xyz cs:y=0.15pt,z=\coo);
}
%\draw (xyz cs:y=3,z=2);
\end{tikzpicture}
\end{center}


\begin{align*}
\begin{bmatrix*}[c]1&0&0\\0&1&0\\0&0&1\end{bmatrix*} \rightarrow \begin{bmatrix*}[c]1&0&0\\0&1&0\\0&0&-1\end{bmatrix*}
\end{align*}

\subsection{Rotation along the y-axis}

\begin{align*}
\begin{bmatrix*}[c]1&0&0\\0&1&0\\0&0&1\end{bmatrix*} \rightarrow \begin{bmatrix*}[c]\cos\theta&0&-\sin\theta\\0&1&0\\\sin\theta&0&\cos\theta\end{bmatrix*}
\end{align*}
\subsection{Rotation along the z-axis}

\begin{align*}
\begin{bmatrix*}[c]1&0&0\\0&1&0\\0&0&1\end{bmatrix*} \rightarrow \begin{bmatrix*}[c]\cos\theta&-\sin\theta&0\\\sin\theta&\cos\theta&0\\0&0&1\end{bmatrix*}
\end{align*}
\subsection{Rotation along the z-axis}

\begin{align*}
\begin{bmatrix*}[c]1&0&0\\0&1&0\\0&0&1\end{bmatrix*} \rightarrow \begin{bmatrix*}[c]1&0&0\\0&\cos\theta&-\sin\theta\\0&\sin\theta&\cos\theta\end{bmatrix*}
\end{align*}
\subsection{Enlargement}
When we enlarge (or conversely, reduce) by scale factor $n$, the unit base vector is multiplied by $n$.\\
Thus the matrix representing an enlargement by scale factor n is given by:
\begin{align*}
n\identitymatrix = \begin{bmatrix*}[c]n&0&0\\0&n&0\\0&0&n\end{bmatrix*}
\end{align*}
\section{Geometric Interpretation of the Determinant}
Consider an object with volume $V$ transformed by the matrix $\mathbf A$ with determinant $|\mathbf A|$. 
The volume of the image is given by $|\mathbf A|V$. Thus, the determinant of the transformation matrix denotes the number of times by which th volume of the object increases or decreases.
\begin{example}
	Describe the effect on volume in $3$D space of the transformation given by $\mathbf{A} = \begin{bmatrix*}[r]1&2&3\\5&0&-1\\2&4&-3\end{bmatrix*}$
\end{example}

\begin{example}
	Find the images of $P(4,5,1)$, $Q(3,-1,-2)$, $R(6,-2,0)$ under the transformation given by \\$\mathbf A = \left(\begin{smallmatrix*}[r]4&3&2\\-1&5&0\\6&2&-3\end{smallmatrix*}\right)$
\end{example}
\begin{example}
	Find the equation of the line which is the image of the line $\mathbf r = 3\mathbf i +\mathbf j +\mathbf k + \lambda(2\mathbf i + \mathbf j - 5\mathbf k)$ under the transformation given by $\mathbf A =\begin{bmatrix}2&3&1\\-1&2&4\\0&6&1\end{bmatrix}$
\end{example}
Find two points on the line.
Find images of points.
Find equation of the image line

\begin{example}
	Find the image of the plane $x+2y-7z=2$ under the transformation defined by $\mathbf A = \begin{smallmatrix}-1&2&1\\-3&1&4\\0&1&2\end{smallmatrix}$
\end{example}
\end{document}

