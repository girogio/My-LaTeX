\documentclass{standalone}
\begin{document}
	\chapter{Surds}
	\section{Introduction}
	\quad Consider numbers $\sqrt{64}, \sqrt{16}$. These can be represented as exact quantities by writing $8$ and $4$. There are however other numbers which cannot be expressed as exact quantities using other symbols.\\
	
	
	There is an option of expressing them as corrected decimals without however preserving their full value. Instead, we choose to keep the form $\sqrt{a}$ which preserves the full value of the numbers.
	\section{Examples}
	\begin{multicols}{2}
		\noindent
		\begin{alignat*}{2}
			a)\quad
			\sqrt{2} & = \sqrt{16\times3}          \\
			& = \sqrt{16} \times \sqrt{3} \\
			& = 4\sqrt{3}                 
		\end{alignat*}
		\begin{alignat*}{2}
			b)\quad
			\sqrt{72} & = \sqrt{8\times9}         \\
			& = \sqrt{9}\times \sqrt{8} \\
			& = 3\sqrt{8}               
		\end{alignat*}
		\begin{alignat*}{2}
			\quad
			\sqrt{360} & = \sqrt{180} \times \sqrt{2} \\
			& =\sqrt{36}\times\sqrt{10}    \\
			& =6\sqrt{10}                  
		\end{alignat*}
		\begin{alignat*}{2}
			d)\quad
			& \quad\left(1+2\sqrt{3}\right)\left(2+3\sqrt{5}\right) \\
			& = 2-\sqrt{3}  - 10\sqrt{3}                            \\
			& =-28-\sqrt{3}                                         
		\end{alignat*}
		\begin{alignat*}{2}
			e)\footnote[1]\quad
			& \quad\left(2-3\sqrt{5}\right)\left(2+3\sqrt{5}\right) \\
			& = 4+6\sqrt{5}-6\sqrt{5}-9(5)                          \\
			& =-41                                                  
		\end{alignat*}
	\end{multicols}
	\footnotetext[1]{This expression shows that for products of the form $\left(a+b\sqrt{c}\right)\left(a-b\sqrt{c}\right)$ the surds will vanish.}
	\newpage
	\section{Rationalizing the denominator}
	\quad \emph{Consider the fraction:} 
	$$\frac{1}{1+\sqrt{2}}$$
	This fraction contains a surd, thus, making it irrational. To rationalize said fraction one should find the multiplicative operation of canceling the denominator \textit{(see \footnotemark[1])}.\\
	
	
	\emph{Continuing...}
	\begin{alignat*}{2}
		\frac{1}{1+\sqrt{2}} & = \frac{1}{1+\sqrt{2}} \times \frac{1-\sqrt{2}}{1-\sqrt{2}} \\
		& =\frac{1-\sqrt{2}}{1}                                       \\
		& =1-\sqrt{2}                                                 
	\end{alignat*} 
\end{document}