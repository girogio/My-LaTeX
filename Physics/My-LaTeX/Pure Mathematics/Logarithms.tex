\documentclass{standalone}
\begin{document}
		\chapter{Logarithms}
	\section{Definition}
	\quad In mathematics, the logarithm is the inverse function to exponentiation. That means the logarithm of a given number $x$ is the exponent to which another fixed number, the base $b$, must be raised, to produce that number $x$.\\
	
	\emph{Consider: }\\$$ 2^3 = 8$$\\
	
	3 is the exponent by which 2 must be raised to obtain 8. This statement can also be reversed:\\ 3 is the logarithm by which with a base of 2, results in 8. Thus:$$ 3 = \log_28$$\\
	
	\emph{In general: }\\$$a^b = c \iff \log_ac = b ,\quad a \in \mathbb{R^+}$$\\
	
	
	Furthermore, it is standard to represent $\log_{10}(x)$ as $\log(10) $ and $ \log_e(x) $ as $ \ln(x) $
	
	
	\bigskip
	\bigskip
	\bigskip
	\bigskip
	
	\tcbset{
		enhanced,
		colback=red!5!white,
		boxrule=0.1pt,
		colframe=black!75!black,
		fonttitle=\bfseries
	}
	\begin{center}
		\begin{tcolorbox}[center title,hbox,    %%<<---- here
			lifted shadow={1mm}{-2mm}{3mm}{0.1mm}%
			{black!50!white}]
			\begin{varwidth}{\textwidth}
				\begin{center}
					$	\log_a1                         = 0                    $ \\
					\bigskip
					$\log_aa                          = 1                    $ \\
					\bigskip
					$	\log_c(ab)                       \equiv \log_ca + log_cb $\\
					\bigskip
					$	\log_c\left (\frac{a}{b}\right)  \equiv \log_ca - log_cb$ \\
					\bigskip
					$	n\log_ca                         \equiv \log_ca^n   $     
				\end{center}
			\end{varwidth}
		\end{tcolorbox} 
	\end{center}
	\section{Proofs}
	
	
	
	\paragraph{Proof 1: }$\log_aa  = 1$
	\begin{center}
		$	\text{Let} \log_aa = x$
		$$   a^x      = a $$
		$$x        =1 $$
	\end{center}
	\paragraph{Proof 2: }$\log_aa  = 1$
	\begin{center}
		$	\text{Let} \log_a1 = x$
		$$   a^x      = 1 $$
		$$x        = 0$$
	\end{center}
	\paragraph{Proof 3: }$\log_cab  = \log_ca + \log_cb$
	\begin{center}
		$	\text{Let} \log_ca = x$ ; $\text{Let} \log_cb = y$
		$$\Rightarrow  c^{x} = a  \text{ ; } c^{y} = b$$
		$$\Rightarrow ab = c $$
		$$\Rightarrow\log_c(ab) = \log_c(x) + \log_c(y)$$
		$$\therefore\quad \log_c(ab) = \log_c(a) + \log_c(b)$$
	\end{center}
	\paragraph{Proof 4: }$\log_c\frac{a}{b}  = \log_ca - \log_cb$
	\begin{center}
		$	\text{Let} \log_ca = x$ ; $\text{Let} \log_cb = y$
		$$  c^{x} = a  \text{ ; } c^{y} = b$$
		$$\frac{a}{b} = c^x \times c^{-y} $$
		$$\log_c\frac{a}{b} = \log_c(x) - \log_c(y)$$
	\end{center}
	\paragraph{Proof 5: }$\log_ca^n = n\log_ca$
	\begin{center}
		$	\text{Let} \log_ca^n = x$
		$$ c^x = a^n$$
		$$c^{\frac{x}{n}} = a  $$
		$$\log_ca = \frac{x}{n}$$
		$$x = n\log_ca $$
	\end{center}
\end{document}