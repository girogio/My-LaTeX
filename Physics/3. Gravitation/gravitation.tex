\documentclass[11pt]{article}
\usepackage[
top    = 2.50cm,% presumably you don't want it to be 0pt as well?
bottom = 2.50cm,
left   = 2cm,
right  = 2cm,
marginparsep = 0pt,
marginparwidth=0pt,
]{geometry}

\usepackage{amssymb}
\usepackage{fancyhdr}
\usepackage[most]{tcolorbox}
\usetikzlibrary{calc}
\usepackage{siunitx}
\usepackage{amsmath}
\usepackage{tikz, tikz-3dplot}
\usepackage{caption}
\usepackage{pgfplots}
\usepackage{multicol}
\pagestyle{fancy}
\fancyhead[l]{Gravitation - Abridged edition}
\fancyhead[r]{Giorgio Grigolo {\textcopyright}}
\newcommand\textcenter[1]{
	\begin{center}
		#1
	\end{center}
}
\begin{document}
	
\section{Newton's Law of Gravitation}

\textcenter{\textit{Any two masses in the universe attract  one another with a force that is proportional 
	to the product of the two masses and inversely proportional to the square of the distance between them.}}

\begin{equation}
	F= \frac{GMm}{r^2}\tag{\si\newton}
\end{equation}

\textcenter{Where G is the \textbf{Universal Gravitational Constant}\footnote{$6.67\times10^{-11}\,\si{\newton\meter\squared\per\kilogram\squared}$}, $M$ and $m$ are two \textbf{masses} (\si{\kilogram}) of two different bodies and $r$ the \textbf{radius} (\si{\meter}) between the primary and secondary bodies respectively.}

\section{Gravitational field strength: }

\begin{equation}
	g=\frac{GM}{R^2}\tag{\si{\newton\per\kilogram}} 
\end{equation}

\textcenter{Where G is the \textbf{Universal Gravitational Constant}\footnotemark[1], $ M $ is the \textbf{mass} (\si{\kilogram}) of a body and $ R $ the \textbf{distance} (\si{\meter}) between the center of the body to the point where the field strength is to be measured\footnote{Most of the time this is the radius of a planet, where the field strength is measured at its surface.}. } 

\section{Velocity of Orbit: }
\begin{equation}
	v=\sqrt{\frac{GM}{r}}\tag{\si{\meter\per\second}}
\end{equation}
\begin{center}
		\tdplotsetmaincoords{70}{110}
	\begin{tikzpicture}[tdplot_main_coords,scale=6]
		\pgfmathsetmacro{\r}{.8}
		\pgfmathsetmacro{\O}{45} % right ascension of ascending node [deg]
		\pgfmathsetmacro{\i}{30} % inclination [deg]
		\draw[<->] (0,-0.035,0) -- node[below]{r} ++ (0,-0.75,0) ;
		
		\tdplotdrawarc[dashed]{(0,0,0)}{\r}{0}{360}{}{}
		
		\tdplotsetrotatedcoords{\O}{0}{0}
		
		\node[circle,fill, inner sep=2pt, label=$m$] (c) at (0,-0.8,0) {};
		\node[circle,fill, inner sep=4pt, label=$M$] (c) at (0,0,0) {};
	\end{tikzpicture}\\
\textcenter{Where G is the \textbf{Universal Gravitational Constant}\footnotemark[1], $ M $ is the \textbf{mass} (\si{\kilogram}) of the priamry body.}
\end{center}

\section{Kepler's Third Law}

\begin{equation}
	T^2 = \frac{4\pi^2r^3}{GM}
\end{equation}

\textcenter{Where $T$ is the \textbf{time of orbit} (\si{\second}), $r$ is the \textbf{distance} (\si\meter) between the center primary body and the center of the secondary body.}
\end{document}