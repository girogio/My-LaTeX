\documentclass[11pt]{article}
\usepackage[
top    = 2.50cm,% presumably you don't want it to be 0pt as well?
bottom = 2.50cm,
left   = 2cm,
right  = 2cm,
marginparsep = 0pt,
marginparwidth=0pt,
]{geometry}

\usepackage{amssymb}
\usepackage{fancyhdr}
\usepackage[most]{tcolorbox}
\usetikzlibrary{calc}
\usepackage{siunitx}
\usepackage{amsmath}
\usepackage{tikz}
\usepackage{caption}
\usepackage{pgfplots}
\usepackage{multicol}
\pagestyle{fancy}
\fancyhead[l]{Work, Energy and Power - Abridged edition}
\fancyhead[r]{Giorgio G.}
\newcommand\textcenter[1]{
	\begin{center}
	#1
	\end{center}
}
\begin{document}


\section{Work: }

\begin{center}
	\begin{equation}
		W = F\times s\tag{\si{\kilogram\meter\squared\per\second\squared}}
	\end{equation}
\textcenter{Where $F$ is a \textbf{force} (\si{\newton})  exerted along a \textbf{distance} (\si{\meter}) $ s $.}
\end{center}


\section{Change in Potential Energy: }

	\begin{equation}
		\Delta G = mgh\tag{\si{\joule}}
	\end{equation}
\textcenter{Where $m$ is an object's \textbf{mass} (\si{\kilogram}), $g$ the \textbf{acceleration} (\si{\meter\per\second\squared}) due to gravity and $h$ the \textbf{height} (\si{\meter}).}

\section{Kinetic Energy: }
\begin{equation}
	E_k = \frac{1}{2} mv^2\tag{\si{\joule}}
\end{equation}

\textcenter{Where $m$ is an object's \textbf{mass} (\si{\kilogram}) and $v$ its \textbf{velocity} (\si{\meter\per\second}).}

\section{Power: }

\begin{multicols}{2}
	\begin{equation}
	P = \frac{E_c}{t}\tag{\si{\watt}}
\end{equation}

\textcenter{Where $E_c$ is the \textbf{energy converted} (\si{\joule}) a nd $t$ the \textbf{time of conversion} (\si\second).}

\begin{equation}
	P = Fv\tag{\si{\watt}}
\end{equation}
\textcenter{Where $F$ is the \textbf{tractive force} (\si{\newton}) and $v$ the \textbf{velocity} (\si{\meter\per\second }).}
\end{multicols}
 \section{Inclined plane: }
		\begin{multicols}{2}
		\begin{center}
			\underline{\textbf{Parallel}}
		\end{center}
		\begin{equation}
		F=mg\sin\theta\tag{\si{\newton}}
		\end{equation}
		\begin{center}
			\underline{\textbf{Perpendicular}}
		\end{center}
		\begin{equation}
		F=mg\cos\theta\tag{\si{\newton}}
		\end{equation}
		
\end{multicols}
\end{document}