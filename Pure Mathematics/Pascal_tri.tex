\documentclass{standalone}
\begin{document}
	\chapter{Pascal's Triangle}
	\bigskip
	\emph{Consider the following expansions:}
	\begin{gather*}
		(1+x)^0 = 1\\
		(1+x)^1 = 1+x\\
		(1+x)^2 = 1+2x+x^2\\
		(1+x)^3 = 1+3x+3x^2+x^3\\
		(1+x)^4 = 1+4x+6x^2+4x^3+x^4\\
		(1+x)^5 = 1+5x+10x^2+10x^3+5x^4+x^5\\
		(1+x)^6 = 1+6x+15x^2+20x^3+15x^4+6x^5+x^6\\
		\ldots\\
		\begin{tabular}{>{$n=}l<{$\hspace{12pt}}*{13}{c}}
			&   &   &   &   &    &    & 1  &    &    &   &   &   &   \\
			&   &   &   &   &    & 1  &    & 1  &    &   &   &   &   \\
			&   &   &   &   & 1  &    & 2  &    & 1  &   &   &   &   \\
			&   &   &   & 1 &    & 3  &    & 3  &    & 1 &   &   &   \\
			&   &   & 1 &   & 4  &    & 6  &    & 4  &   & 1 &   &   \\
			&   & 1 &   & 5 &    & 10 &    & 10 &    & 5 &   & 1 &   \\
			6 & 1 &   & 6 &   & 15 &    & 20 &    & 15 &   & 6 &   & 1 
		\end{tabular}
	\end{gather*}\\
	The above array of numbers is called Pascal's Triangle. It can be used to expand any binomial. The following examples illustrate its use.\\
	\newpage
	\begin{example}
		Expand the following using Pascal's Triangle
	\end{example}
	\begin{alignat*}{2}
		(1+2x)^5                      & \equiv 1(2x)^0 + 5(2x)^1 + 10(2x)^2 + 10(2x)^3 + 5(2x)^4 + 2x^5\tag*{a)}                                                                          \\
		& \equiv 1 + 10x + 40x^2 + 80x^3 + 80x^4 + 32x^5                                                                                                    \\
		\\	
		\left(1-\frac{3x}{2}\right)^6 & \equiv 1\left(\frac{-3x}{2}\right)^0 + 6\left(\frac{-3x}{2}\right)^1 + 15\left(\frac{-3x}{2}\right)^2 + 20\left(\frac{-3x}{2}\right)^3 +\tag*{b)} \\
		& \quad 15\left(\frac{-3x}{2}\right)^4 + 6\left(\frac{-3x}{2}\right)^5 + 1\left(\frac{-3x}{2}\right)^6                                              \\
		& \equiv 1 - 9x + \frac{135x^2}{4} - \frac{135x^3}{2} + \frac{1215x^4}{16} - \frac{243x^5}{16} + \frac{728x^6}{64}                                  \\
		&   \\(p+q)^4 &\equiv (p(1+\frac{q}{p}))^4\tag*{c)}\\
		& \equiv p^4\left(1+\frac{q}{p}\right)^4                                                                                                            \\
		& \equiv p^4\left(1 + \frac{4q}{p} + \frac{6q^2}{p^2} + \frac{4q^3}{p^3} + \frac{q^4}{p^4}\right)                                                   \\
		& \equiv p^4 + 4p^3q + 6p^2q^2 + 4pq^3 + q^4                                                                                                        
	\end{alignat*}
	In the above examples we observe that: 
	
	\begin{itemize}
		\item{The expansion contains the coefficients for Pascal's Triangle.}
		\item{The expansion is formed by descending exponents of the first term of the binomial \& ascending exponents of the second.}
		\item{The sum of the exponents in each term is equal to the 	exponent by which the binomial was raised.}
	\end{itemize}
	These observations can be applied to expand any binomials raised with positive integer exponents.	
\end{document}