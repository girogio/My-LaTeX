	\documentclass{standalone}
	\begin{document}
	\chapter{Partial Fractions}
	\section{Introduction}
	\emph{Consider the expression and suppose it is simplified:}
	
	
	\begin{alignat*}{2}
		\frac{2}{x+1}+\frac{3}{2x-5} & = \frac{2(2x-5)+3(x+1)}{(x+1)(2x-5)} \\
		& =\frac{4x-10+3x+3}{(x+1)(2x-5)}      \\
		& =\frac{4x-10+3x+3}{(x+1)(2x-5)}      \\
		& =\frac{7x-7}{(x+1)(2x-5)}            \\
	\end{alignat*}
	
	In this chapter we reverse the approach above, hence decomposing one fraction to its corresponding partial fractions.
	\newpage
	\section{Types of partial fraction cases}
	\subsection*{\underline{Type 1: Linear Factors in denominator}}
	\begin{example}
		Decompose $ \frac{7x-7}{\left(x+1\right)}$ into Partial Fractions
	\end{example}
	\begin{alignat*}{2}	
		&            & \frac{7x-7}{\left(x+1\right)} & \equiv \frac{A}{\left(x+1\right)} + \frac{B}{\left(2x-5\right)} \\
		& \implies   & 7x-7                          & = A\left(2x-5\right)+B\left(x+1\right))                         \\	
		& \implies   & 7(-1)-7                       & = A(2(-1)-5)\tag*{$x = -1$}                                     \\
		& \implies   & -14                           & = -7A                                                           \\\leqnomode
		& \implies   & A                             & = 2\tag{..1}                                                    \\
		& \implies   & 7\left(\frac{5}{2}\right)-7   & =B\left(\left(\frac{5}{2}\right)+1\right)\tag*{$x=\frac{5}{2}$} \\
		& \implies   & \frac{35}{2} - 7              & = \frac{7B}{2}                                                  \\
		& \implies   & B                             & =3\tag{..2}                                                     \\\\
		& \therefore & \frac{7x-7}{\left(x+1\right)} & = \frac{2}{\left(x+1\right)} + \frac{3}{\left(2x-5\right)}      
	\end{alignat*}
	
	
	\subsection*{\underline{Type 2: Irreducible Quadratic Factor in Denominator}}
	\begin{example}
		Decompose $\frac{x^2 +1}{(2x+1)(x^2 +3)}$ into its corresponding partial fractions.
	\end{example}
	\begin{alignat*}{2}
		&            & \frac{x^2 +1}{(2x+1)(x^2 +3)} & \equiv \frac{A}{2x+1}+\frac{Bx+C}{x^2 +3}       \\
		& \implies   & x^2 + 1                       & \equiv A(x^ 2 + 3) + (Bx+C)(2x+1)               \\
		& \implies   & x^2 + 1                       & \equiv x^2(A+2B) + x(2B + 2C) + (3A+C)\tag{*..} \\
		\intertext{\quad At this stage, since both equations are identical, we analyse the different coefficients and constants to form a system of equations to solve.}
		\intertext{Comparing coefficients of $x^2\colon$}
		&            & 1                             & = A+2B\tag{1..}                                 \\
		\intertext{Comparing coefficients of $x\colon$}
		&            & 0                             & = B+C\tag{2..}                                  \\
		\intertext{Comparing constants$\colon$}
		&            & C                             & = 1-3A\tag{3..}                                 \\
		\intertext{Substituting  3.. in 2..}
		&            & 0                             & =B+1-3A                                         \\
		& \implies   & B                             & =3A-1\tag{4..}                                  \\
		\intertext{Substituting 4.. in 1..}
		&            & 1                             & = A + 2(3A-1)                                   \\
		& \implies   & A                             & = \frac{-1}{7} \tag{5..}                        \\
		\intertext{Substituting 5.. in 1..}
		&            & 1                             & = \frac{-1}{7} +2B                              \\
		& \implies   & -7                            & = 1-14B                                         \\
		& \implies   & B                             & =\frac{4}{7}\tag{6..}                           \\
		\intertext{Substituting 5.. in 3..}
		&            & C                             & = 1-3\left(\frac{-1}{7}\right)                  \\
		& \implies   & C                             & = \frac{10}{7}                                  \\
		& \therefore & \frac{x^2 +1}{(2x+1)(x^2 +3)} & \equiv \frac{A}{2x+1}+\frac{Bx+C}{x^2 +3}       
	\end{alignat*}
	\newpage
	\subsection*{\underline{Type 3: Repeated factor in the denominator}}
	\begin{example}
		Decompose $\frac{x+1}{(x+2)(x-3)^2}$ into its corresponding partial fractions.
	\end{example}
	\begin{alignat*}{2}
		&            & \frac{x+1}{(x+2)(x-3)^2} & \equiv \frac{A}{x+2} + \frac{B}{x-3} + \frac{C}{(x-3)^2}           \\
		& \implies   & x+1                      & \equiv A(x-3)^2 + B(x-3)(x+2) + C(x+2)                             \\
		& \implies   & x+1                      & \equiv Ax^2 - 6Ax +9A + Bx^2-Bx-6B + Cx+2C)                        \\
		& \implies   & -2+1                     & =A(-2-3)^2\tag*{$x=-2$}                                            \\
		& \implies   & A                        & = \frac{-1}{25}                                                    \\
		& \implies   & 3+1                      & =C(3+2)\tag*{$x=3$}                                                \\
		& \implies   & C                        & =\frac{4}{5}                                                       \\
		\intertext{Comparing coefficients of $x^2\colon$}
		& \implies   & 0                        & = A + B                                                            \\
		& \implies   & B                        & = \frac{1}{25}                                                     \\
		& \therefore & \frac{x+1}{(x+2)(x-3)^2} & \equiv \frac{1}{25(x-3)} + \frac{4}{5(x-3)^2} - \frac{-1}{25(x-3)} \\
	\end{alignat*}
	
	The approach above, is similar to the previous one, with the addition of the fact that each repeated factor has to be listed in order of powers until its own.
	\newpage
	\subsection*{\underline{Type 4: Improper fraction}}
	\begin{example}
		Decompose $\frac{2x^2 -8x +11}{2x-5}$ into its corresponding partial fractions.
	\end{example}
	Since the fraction is improper, or top-heavy\footnotemark[2] it is required to perform a polynomial long division and acquire the proper terms. \footnotetext[2]{Improper fractions containing a variable are recognized by the order of the exponent when the expression is expanded. i.e $\frac{x^2}{x+5}$ is regarded as improper}
	\begin{center}
		\polylongdiv{2x^2 -8x +11}{2x-5}
	\end{center}
	\begin{alignat*}{2}
		& \therefore\quad & \frac{2x^2 -8x +11}{2x-5} & \equiv 	x - \frac{3}{2} + \frac{7}{2(2x-5)} 
	\end{alignat*}
	\newpage
	\end{document}