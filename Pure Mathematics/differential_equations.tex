\documentclass{standalone}


\begin{document}
	
	\chapter{Differential Equations}
	
	\section{First order differential equations}
	\subsection{Exact differential equations}
	When solving \textit{exact differential equations} we recognize that one side of the equation is of the form $u\frac{dv}{dx} + v\frac{du}{dx}$ and hence  $\int u\frac{dv}{dx} + v\frac{du}{dx}\,dx$ can be quoted as $uv+k$.
	
	\subsection{Integrating factor}
	Consider a first order differential equation that can be written in the form $\frac{dy}{dx} + f(x)y + g(x)$. The LHS of the differential equation ia not yet exact, but suppose that it becomes exact when it is multiplied by a function $I(x)$.\\
	
	Thus, we have the exact differential equation: \[I(x)\left(\frac{dy}{dx} + f(x)y =  g(x)\right)\]
	
	Which is simplified to: \[I(x)\frac{dy}{dx} + I(x)f(x)y = I(x)g(x)\]
	
	Comparing the LHS with $v\frac{du}{dx} + u\frac{dv}{dx}$ we have: 
	\begin{align*}  v&=I &\quad u&=y\\    \frac{du}{dx} &= I(x)f(x) &\quad \frac{dv}{dx} &= \frac{dy}{dx} \end{align*}
	
	Which results in: 
	\begin{align*}
		\int\frac{1}{I(x)}\,dI(x)      & = \int f(x)\,dx         \\
		\implies \ln\left| I(x)\right| & = \int f(x)\, dx        \\
		\therefore \quad	I(x)          & = e^{\int f(x)\,dx} \qed 
	\end{align*}
	
	\newpage
	\section{Second order differential equations}
	A second order differential equation is one of the form: \[a\frac{d^2y}{dx^2} + b\frac{dy}{dx} + cy= f(x)\] with a general solution: \[y=\text{C.F.} + \text{P.I.}\]
	
	\subsection{The complementary function}
	To obtain this part of the solution, the general solution of the quadratic equation $ax^2 + bx + c=0$ by comparing it to $a\frac{d^2y}{dx^2} + b\frac{dy}{dx} + cy = 0$ when $f(x)$, or the \textbf{RHS} $= 0$, and thus the two so called roots can be found. 
	
	\begin{center}
		\renewcommand{\arraystretch}{1.6}
		\begin{tabular}{|c|c|}
			\hline
			Roots                                        & General Solution                 \\ \hline
			Two real distinct roots $\alpha$ and $\beta$ & $y=Ae^{\alpha x} + Be^{\beta x}$ \\ \hline
			Two real equal roots $\alpha$ and $\alpha$   & $y=(A+Bx)e^{\alpha x}$           \\ \hline
			Two complex roots $p\pm qi$                  & $y=e^{px}(A\cos qx + B\sin qx)$  \\\hline
		\end{tabular}
	\end{center}	
	
	\subsection{Type 2: $\mathbf{f(x) \neq 0}$}
	When $f(x) \neq 0$ the general solution of such a differential equation is of the form $y=\text{C.F. + P.I.}$, where C.F. is the complimentary function and P.I. is the particular integral.

	The C.F. is obtained by finding the general solution of the differential equation $a\frac{d^2y}{dx^2} + b\frac{dy}{dx} + cy = 0$, similarly to the previous case.\\

	The P.I. is any solution of the given differential equation. It depends on the function $f(x)$ and is usually a general form of it.
	\begin{center}
		\renewcommand{\arraystretch}{1.6}
		\begin{tabular}{|c|c|}
			\hline
			$\mathbf{f(x)}$		& \textbf{Trial Solution}\\\hline
			$5$					& $y=k$\\\hline
			$2x+7$				& $y=px+q$\\\hline
			$3x^2+x-6$			& $y=px^2+qx+r$\\\hline
			$3e{7x}$			& $y=ke^{7x}$\\\hline
			$3xe^{-2x}$     	& $y=(px+q)e^{-2x}$\\\hline
			$2\sin3x + 4\cos3x$ & $y=p\sin3x + q\cos3x$\\\hline
			$\cos8x$ 			& $y=p\sin8x + q\cos8x$\\\hline
			$\sin x - 3\cos2x$  & $y=p\sin x + q\cos x +r\sin2x + t\sin2x$\\\hline
			$2x + 6e^{4x}$		& $y=px+q+ke^{4x}$\\\hline
		\end{tabular}
	\end{center}
	\newpage
	
	The trial solution is to be chosen according to the above table and differentiated twice. These values are then substitued in the given differential equation, so that the unknown constant/s of the trial solution can be found.\\
	
	The general solution $y=\text{C.F.} + \text{P.I.}$ is then written.\\
	
	The unknown constants of the C.F. can be found if additional information is given  (i.e. $y=1$ and $\frac{dy}{dx}=0$ when $x=0$).\\
	
	In some cases (failure cases) the trial solution listed in the above table leads to an \textit{inconsistent equation}. This usually happens when the trial solution is included in the complementary function. The correct trial solution is obtained by multiplying the trial solution in the above table by $x$ or $x^2$. Such trial solutions are usually given by the question. 
	
\end{document}