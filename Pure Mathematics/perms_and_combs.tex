\newcommand*{\perm}[1][-3mu]{\permcomb[#1]{P}}
\newcommand*{\permcomb}[4][0mu]{{{}^{#3}\mkern#1#2_{#4}}}


\begin{document}
	\chapter{Permutations and Combinations}
	Consider $n$ objects from which $r$ are to be arranged in a particular order, where $r \leq n$. The number of \textit{permutations} of which $r$ objects from a total of $n$ refers to the number of ways in which these $r$ objects can be arranged, where the order of arrangement matters.\\
	
	
	Let us consider $3$ letters \{A, B, C\}. There are $6$ possible arrangements, or \textit{permutations}, of these letters, namely:
	\begin{center}
		\set{\set{A,B,C}, \set{A,C,B}, \set{B,A,C}, \set{B,C,A}, \set{C,B,A}, \set{C,A,B}}
	\end{center}
	In general, the number of permutations of $r$ objects from a total of $n$ is denoted by $\perm n r$ defined as \[\perm n r = \dfrac{n!}{(n-r)!}\]
	
	\begin{example}
		Consider the set of letters $\set{A,B,C,D,E}$. 
		
		\quad\textbf{(a)}How many of these arrangements start with a vowel?
	\end{example}
	
	\begin{example}
		Consider the set of numbers $\set{1,2,3,4,5,6}$.
		
		\quad\textbf{(a)} In how many ways can a $4$ digit number be formed from the above set? 
		
		\quad\textbf{(b)} How many of these numbers are even? 
		
		\quad\textbf{(c)} How many of these numbers are greater than 3000?
	\end{example}
	
	\begin{example}
		Consider the set of numbers $\set{1,2,3,4,5,0}$.
		
		\quad\textbf{(a)} How many $3$ digit numbers can be formed?
		
		\quad\textbf{(b)} How many of these numbers are even?
		
		\quad\textbf{(c)} How many of these numbers are greater than $400$?
		
		\quad\textbf{(d)} How many even numbers can be formed?
	\end{example} 
	
	\begin{align*}
	{\textbf{a.}} \quad & 5\times5\times4 & = 100 \\
	{\textbf{b.}} \quad & 5\times4\times1 & = 20  \\
	{\textbf{c.}} \quad & 5\times5\times4 & = 100 \\
	{\textbf{d.}} \quad & 5\times5\times4 & = 100 
	\end{align*}
	\section{Permutations with Identical Objects}
	The above method, however, does not suffice in the case that we have identical objects. Suppose we have the set $\set{S,E_1,E_2}$. For every time $t$ the repeated element is present in the given set, we have to divide the total we have to divide by $t!$
	\begin{example}
		In how many ways can the letters of the word `MALTA' be arranged?
	\end{example}
	\[\frac{5!}{2!}\]
	
	\begin{example}
		In how many ways can the letters of the word `ILLUSTRATIONS' be arranged?
	\end{example}
Since the letters $\set{I,L,S,T}$ are repeated twice, the total possibilities have to be divided by the factorial of the number of each recurring letter. (\textit{i.e., divide by $2!$ for the two `I's, by $2!$ for the two `L's ...})
	\[\frac{13!}{2!\,2!\,2!\,2!}\]
	\section{Circular Permutations}
	In a particular field of mathematics referred to as group theory, a cyclic permutation is a permutation of the elements of some set $X$ which maps the elements of some subset $S$ of $X$ to each other in a cyclic fashion, while fixing all other elements of $X$. In other words, this is the number of ways in which a set can be permuted whilst omitting identical cycles.
	\begin{example}
		In how many ways can 6 people be seated at a round table?
	\end{example}

\begin{example}
	In how many ways can 4 couples be arranged around a table?
	
	\quad\textbf{(a)} In  how many of these arrangements are all the males separated? (\textit{i.e., no male sits next to another})
	
	
\end{example}
\end{document}